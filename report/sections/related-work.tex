\subsection{Knowledge Distillation}\label{subsec:rw:knowledge-distillation}
Hinton et al. proposed the concept of knowledge distillation~\cite{hinton2015distilling},
providing the basis for model compression and optimization.
Knowledge distillation uses soft targets of the teacher model as the learning target for the smaller student model
by minimizing the KL divergence between the output probability distributions of the teacher and the student.
After that, FitNets~\cite{romero2014fitnets} proposed using intermediate representations,
such as feature maps and hidden states, of the teacher model as a guidance for students to learn internal representations.
Students not only imitate the final outputs of the teachers, but also replicate their details.
Then, DistilBERT~\cite{sanh2019distilbert} applied this technique to BERT, a Transformer model,
reducing its parameters by 40\% percent while retaining more than 97\% of the performance on various workloads.
Subsequently, Shleifer and Rush~\cite{shleifer2020pre} extended knowledge distillation techniques to the field of text summarizations.
Their approach introduced task-related techniques to distill large summarization models,
such as T5 and BART, into smaller models, with a smaller model inference time.
These researches demonstrate that by effectively extracting and transferring knowledge from different layers of the teacher model,
knowledge distillation can significantly reduce model footprint and inference time,
making models adaptable to various application scenarios.

\subsection{Pruning}\label{subsec:rw:pruning}
Early pruning strategy focused on sparsifying weight matrices by removing weights with small values,
which is unstructured pruning~\cite{han2015learning}.
For Transformers, structured pruning is applied to optimize the self-attention and feed-forward neural network components.
This approach calculates importance scores to select necessary attention heads and drop the others~\cite{michel2019sixteen}
without significantly degrading the model performance.
After that, movement pruning~\cite{sanh2020movement} was proposed to dynamically adjust the pruning strategy during training
or fine-tuning process according to gradient information, improving the overall model precision and robustness after pruning.
This strategy can be applied to both unstructured and structured pruning, which is flexible.
Famous real-world model pruning implementations include~\texttt{torch.nn.utils.prune},
which contains both unstructured and structured pruning.
For unstructured pruning, the implementation leverages random masks or weight thresholds to drop weights in a parameter-by-parameter manner.
Structured pruning is a larger-granularity pruning that removes neurons, convolution kernels, or attention heads.

\subsection{Quantization}\label{subsec:rw:quantization}
Quantization includes weight quantization and activation quantization.
In recent years, 8-bit weight and activation quantization has become mainstream in Transformer models,
since quantizing a 32-bit float point Transformer to an 8-bit model can largely reduce the model footprint
and inference cost while retaining high precision.
For instance, Q8BERT~\cite{zafrir2019q8bert} introduced a quantized 8-bit BERT model,
reducing the model size by approximately 4x by applying quantization-aware training.
Furthermore, TernaryBERT~\cite{zhang2020ternarybert} uses a ternary quantization mechanism,
and the weights of the model only take values in \texttt{\{-1, 0, 1\}},
further achieving a higher compression rate while retaining the performance.
Real-world implementations of quantization include TorchQuantization,
which provides implementations of post-training quantization and quantization-aware training.
Post-training quantization requires the offline collection of calibration data to generate precise scale factors
after converting the model.
For quantization-aware training, TorchQuantization supports inserting fake quantization modules
in the model and significantly improves the performance of low-precision models by simulating
and propagating quantized model loss to optimize the weights of the quantized model.
